\documentclass[12 pt]{report}

\usepackage{epsfig}
\usepackage[in,empty]{myfullpage}
\usepackage{amssymb}
\usepackage{amsmath}

\baselineskip=20pt

\begin{document}

\noindent \vfill \noindent \large

\centerline{Math 324 B  - Winter 2017}

\centerline{Midterm exam 2}

\centerline{Friday, February 17, 2017}

\normalsize

\vfill
\medskip
Name: \rule{10cm}{1pt}

\bigskip

\vfill
\begin{center}
{\large
\begin{tabular}{||c|c|r||}
\hline Problem 1 & 10 & \hspace{10mm} \hfill \\
\hline Problem 2 & 14 & \hspace{10mm} \hfill \\
\hline Problem 3 & 16 & \hspace{10mm} \hfill \\
\hline Problem 4 & 10  & \hspace{10mm} \hfill \\
\hline Total & 50 & \hspace{10mm} \hfill \\
\hline
\end{tabular}
}
\end{center}
\vfill
\begin{itemize}
\item There are 4 problems on this exam. Make sure you have all four.
\item You must show your work on all problems.  The correct answer
with no supporting work may result in no credit. \textbf{Put a box
around your FINAL ANSWER for each problem and cross out any work
that you don't want to be graded.} 
\item Give exact answers, and simplify as much as possible. 
For example, $\frac{\pi}{\sqrt{2}}$ is acceptable, but $\frac{1}{2} + \frac{3}{4}$
should be simplified to $\frac{5}{4}$.   

\item If you need more room, use the backs
of the pages and indicate to the grader that you have done so.
\item Raise your hand if you have a question.
\item Any student found engaging in academic misconduct will receive
a score of 0 on this exam.
\item You have 50 minutes to complete the exam.  Budget your time wisely! \\
\end{itemize}
\vfill
\begin{center}GOOD LUCK!\end{center}

\newpage
\begin{enumerate}

\item (10 points) Let $f(x,y,z) = x^2 \sin(z) - \frac{x^2+y^2}{z^2 + 1}$. 

\begin{enumerate} \item Find $\nabla f(x,y,z)$, and evaluate $\nabla{f}(1,-1, 0)$. 

\vfill

\item Find the directional derivative of $f$ in the $y-$direction at the point $(3, 2, -1)$: that is, evaluate $D_{u}f(3, 2, -1)$, where $u = \langle 0, 1, 0 \rangle$. 

\end{enumerate} 

\vfill

\newpage

\item (14 points) Consider the vector field $F = \langle x^2y, y \rangle$, and the closed curve $C$ consisting of the parabola $y=x^2$ for $|x| \leq 1$ and the line segment connecting $(-1,1)$ to $(1,1)$. Give $C$ the clockwise orientation -- so $C$ traverses the parabola from right to left, and the line segment from left to right. Draw a picture of $C$, and use Green's theorem to evaluate $\int_C F \cdot dr$. 

\newpage


\item (16 points) For each of the statements below, circle \textbf{True} if you think it is true, and circle \textbf{False} if you think it is false. You may use the space to do scratch work, but no partial credit will be awarded on this question. (4 points for each statement.)

\begin{enumerate} 
\item \textbf{True} \hspace{5pt} \textbf{False} \hspace{10pt} The vector field $F = \langle x^3, 3x^2 y  \rangle$, defined on all of $\mathbb{R}^2$, is conservative. 

\vfill

\item \textbf{True} \hspace{5pt} \textbf{False} \hspace{10pt} The vector field $F = \frac{y \hat{i} - x \hat{j}}{x^2+y^2}$, defined on $\mathbb{R}^2$ minus the origin, is conservative.

\vfill

\item \textbf{True} \hspace{5pt} \textbf{False} \hspace{10pt} A conservative vector field $F$ defined on any region $D \subset \mathbb{R}^2$ has $\int_C F \cdot dr = 0$ for any closed curve $C$ lying inside $D$. 

\vfill

\item \textbf{True} \hspace{5pt} \textbf{False} \hspace{10pt} Consider the function $F(x,y,z) = 2x^4 + y^4 -z^4$. Let $S$ be the surface defined by the equation $F(x,y,z) = -14$, and consider the point $p = (1,0,2)$. Then $p$ lies on $S$, and the tangent plane to $S$ at the point $p$ has normal vector parallel to $\langle -1, 0, -4 \rangle$. 

\vfill

\end{enumerate}

\newpage


\item (10 points) Let $f(x,y) = 2x + y$, and consider the line segment $C$ starting at $(-2, 4)$ and ending at $(1,3)$. 

\begin{enumerate} \item Evaluate $\int_C \nabla f \cdot dr$ using the fundamental theorem for line integrals. 

\vfill

\item Here's a parameterization of $C$: $r(t) = \langle 3t-2, 4-t \rangle$ for $0 \leq t \leq 1$. Use it to evaluate $\int_C \nabla f \cdot dr$ directly.

\vfill

\end{enumerate}

\end{enumerate}


\end{document}
