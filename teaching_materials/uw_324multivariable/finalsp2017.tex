\documentclass[12 pt]{report}

\usepackage{epsfig}
\usepackage[in,empty]{myfullpage}
\usepackage{amssymb}
\usepackage{amsmath}

\baselineskip=20pt

\begin{document}

\noindent \vfill \noindent \large

\centerline{Math 324 B  - Spring 2017}

\centerline{Final exam}

\centerline{Wednesday, June 7th, 2017}

\normalsize

\vfill
\medskip
Name: \rule{10cm}{1pt}

\bigskip

\vfill
\begin{center}
{\large
\begin{tabular}{||c|c|r||}
\hline Problem 1 & 15 & \hspace{10mm} \hfill \\
\hline Problem 2 & 15 & \hspace{10mm} \hfill \\
\hline Problem 3 & 14 & \hspace{10mm} \hfill \\
\hline Problem 4 & 16  & \hspace{10mm} \hfill \\
\hline Problem 5 & 15  & \hspace{10mm} \hfill \\
\hline Problem 6 &  15 & \hspace{10mm} \hfill \\
\hline Problem 7 & 10 & \hspace{10mm} \hfill \\
\hline Total & 100 & \hspace{10mm} \hfill \\
\hline
\end{tabular}
}
\end{center}
\vfill
\begin{itemize}
\item There are 7 problems (8 pages) in this exam. Make sure you have them all. 
\item You must show your work on all problems.  The correct answer
with no supporting work may result in no credit. \textbf{Put a box
around your FINAL ANSWER for each problem and cross out any work
that you don't want to be graded.} 
\item Give exact answers, and simplify as much as possible. 
For example, $\frac{\pi}{\sqrt{2}}$ is acceptable, but $\frac{1}{2} + \frac{3}{4}$
should be simplified to $\frac{5}{4}$.   

\item If you need more room, use the backs
of the pages and indicate to the grader that you have done so.
\item Raise your hand if you have a question.
\item Any student found engaging in academic misconduct will receive
a score of 0 on this exam.
\item You have 2 hours to complete the exam.  Budget your time wisely! \\
\end{itemize}
\vfill
\begin{center}GOOD LUCK!\end{center}

\newpage
\begin{enumerate}

\item (15 pts) Let $C$ be the cone $4y^2+4z^2 = x^2$ for $0 \leq x \leq 4$, oriented inward (i.e. normal points toward the positive $x-$axis). Use Stokes' theorem to evaluate

\[
\iint_C \nabla \times \langle z^2, 0, 1+xy \rangle \, \cdot dS.
\]

Make sure to indicate how you are orienting the boundary. 

\vfill

\newpage \item (15 pts) Evaluate the surface integral

\[
\iint_S \langle x^2, 2z, y^2 \rangle \, \cdot dS,
\]

where $S$ is the boundary of the quarter sphere region $E = \{(x,y,z): x^2 + y^2 + z^2 \leq 9, x, y \leq 0 \}$ oriented inward, i.e. towards $E$. 

\vfill

\newpage \item (14 pts) Let $D$ be the ellipsoidal cylinder defined by the equation $x^2 + 3z^2 = 4,$ for any $-1 \leq y \leq 1$. 

\begin{enumerate} \item (5 pts) Give a parameterization of $D$ in terms of the coordinates $\theta$ and $y$. 

\vspace{4cm}

\item (4 pts) Compute $r_\theta \times r_y$. 

\vfill

\item (5 pts) Find a vector that is normal to $D$ at the point $(x,y,z) = (1, \sqrt{3}/2, 1)$. 

\vfill

\end{enumerate}

\newpage \item (16 pts; 4 pts each) For each of the statements below, circle \textbf{True} if you think it is true and \textbf{False} if you think it is false. 

\begin{enumerate} \item \textbf{True}\hspace{5pt} \textbf{False} \hspace{5pt} If $F$ is a conservative vector field and $C$ is any curve, then 

\[
\int_C F \, \cdot dr = 0.
\]

\vfill

\item \textbf{True} \hspace{5pt} \textbf{False} \hspace{5pt} Let $S$ denote the surface of the sphere of radius $17$ centered at $(1,0, \sqrt{2})$, oriented inward. For any vector field $F$, 

\[ 
\iint_S \nabla \times F \, \cdot dS = 0.
\]

\vfill

\item \textbf{True} \hspace{5pt} \textbf{False} \hspace{5pt} For any vector field $G$, $\nabla \times (\nabla \times G) = 0$. 

\vfill

\item \textbf{True} \hspace{5pt} \textbf{False} \hspace{5pt} If $R \subset \mathbb{R}^3$ is a region in space, and $S = \partial R$ is the boundary surface of $S$, then 

\[
\iiint_R (2z + 2y) \, dV = \iint_S (2xz + y^2 - 3) \, dS.
\]
\vfill


\end{enumerate}

%\newpage
%
%\item (20 pts) Let $E$ denote the unit cube in $\mathbb{R}^3$, $E = \{(x,y,z): 0 \leq x, y, z \leq 1\}$. Let $S = \partial E$ be the boundary of the cube, with the outward pointing orientation (away from $E$). Use the divergence theorem to evaluate 
%
%\[
%\iint_{\partial E} \langle 0, 0, e^{y+z} \rangle \cdot dS.
%\]

% \item (20 pts) Consider the radial vector field $F = \frac{1}{|r|^4} r$, where $r = \langle x, y, z \rangle.$ 
%
%\begin{enumerate} \item (5 pts) Find the divergence of $F$: write your final answer in terms of $r$. 
%
%\vfill
%
%\item (7 pts) Let $E_R$ be the spherical region $\{(\rho, \theta, \phi): 1 \leq \rho \leq R\}$, where $R > 1$ is a fixed number. Evaluate the volume integral
%
%\[
%\iiint_{E_R} \nabla \cdot F \, dV.
%\]
%
%\vfill
%
%\newpage
%
%\item (5 pts) Let $S_1$ denote the surface of the sphere of radius $1$ oriented toward the origin, and let $S_R$ denote the surface of the sphere of radius $R$ oriented away from the origin. According to the divergence theorem, 
%
%\[
%\iint_{S_1} F \cdot dS + \iint_{S_R} F \cdot dS = \iiint_{E_R} \nabla \cdot F \, \cdot dS.
%\]
%
%You found the value of the right hand side of this equation in part b: compute one of the integrals on the left hand side, and use your answer to find the value of the other integral. 
%
%\vfill 
%
%\item (3 pts) What happens to the values of the three integrals from part $c$ when $R \to \infty$?
%
%\vfill
%
%\end{enumerate}
%

\newpage

\item (12 pts; 4pts each) Consider the vector field $F = \langle 2y + 1, x+y, 0 \rangle$ defined on all of $\mathbb{R}^3$. 

\begin{enumerate} \item Use Green's theorem to compute $\int_C F \cdot dr$, where $C$ is the curve in $\mathbb{R}^3$ parameterized by $r(t) = \cos(t) \hat{i} + \sin(t)\hat{j}$ for $t \in [0,2\pi]$. 

\vfill 

\item Find $\nabla \cdot F$ and $\nabla \times F$. 

\vfill

\item Let $S$ be the surface of the box $[0,3] \times [2,3] \times [-1,1] \subset \mathbb{R}^3$, oriented outward: that is, $S$ is the boundary of the region $\{(x,y,z): 0 \leq x \leq 3, 2 \leq y \leq 3, -1 \leq z \leq 1\}.$ What is $\iint_S F \cdot dS$?

\vfill

\end{enumerate}

\newpage

Let $S$ denote the parabaloid $2x^2 + y + z^2 = 1$. Both problems 6 and 7 are about the surface $S$.

\item (15 pts) Find a point $(x,y,z)$ where the normal vector to $S$ at $(x,y,z)$ is parallel to the vector $\langle 4,1,2 \rangle$. Are there other points where the normal vector is parallel to $\langle 4,1,2 \rangle$? Explain. 

\vfill

\newpage

\item (10 pts) Suppose $S$ represents an infinitely large sheet of charged material, with charge density 

\[
f(x,y,z) = \frac{1}{4\pi \epsilon_0} \frac{e^{-x^2-z^2}}{\sqrt{1+16x^2+4z^2}},
\]

where $\epsilon_0$ is a constant. Compute the total charge in the plate $S$ by evaluating the integral

\[
\iint_S f(x,y,z) \, dS.
\]

\vfill


\end{enumerate}


\end{document}
