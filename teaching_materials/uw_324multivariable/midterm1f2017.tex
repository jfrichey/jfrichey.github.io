\documentclass[12 pt]{report}

\usepackage{epsfig}
\usepackage[in,empty]{myfullpage}
\usepackage{amssymb}
\usepackage{amsmath}

\baselineskip=20pt

\begin{document}

\noindent \vfill \noindent \large

\centerline{Math 324 E - Fall 2017}

\centerline{Midterm exam 1}

\centerline{Wednesday, October 18, 2017}

\normalsize

\vfill
\medskip
Name: \rule{10cm}{1pt}

\bigskip

\vfill
\begin{center}
{\large
\begin{tabular}{||c|c|r||}
\hline Problem 1 & 14 & \hspace{10mm} \hfill \\
\hline Problem 2 & 12  & \hspace{10mm} \hfill \\
\hline Problem 3 & 12 & \hspace{10mm} \hfill \\
\hline Problem 4 & 12  & \hspace{10mm} \hfill \\
%\hline Problem 5 & 10 & \hspace{10mm} \hfill \\
\hline Total & 50 & \hspace{10mm} \hfill \\
\hline
\end{tabular}
}
\end{center}
\vfill
\begin{itemize}
\item There are 4 questions on this exam. Make sure you have all four.
\item \textbf{You must show your work on all problems.}  The correct answer
with no supporting work may result in no credit. Put a box
around your FINAL ANSWER for each problem and cross out any work
that you don't want to be graded.
\item Give exact answers, and simplify as much as possible. 
For example, $\frac{\pi}{\sqrt{2}}$ is acceptable, but $\frac{1}{2} + \frac{3}{4}$
should be simplified to $\frac{5}{4}$.  

\item If you need more room, use the backs
of the pages and indicate to the grader that you have done so.
\item Raise your hand if you have a question.
\item Any student found engaging in academic misconduct will receive
a score of 0 on this exam.
\item You have 50 minutes to complete the exam.  Budget your time wisely! \\
\end{itemize}
\vfill
\begin{center}GOOD LUCK!\end{center}

\newpage
\begin{enumerate}

\item \begin{enumerate} 

\item (7 pts) Let $B \subset \mathbb{R}^3$ be the region inside the sphere $x^2 + y^2 + z^2 = 16$, inside the half space $x \geq 0$, inside the cone $x^2 = 3y^2 + 3z^2$, and outside the sphere $x^2 + y^2 + z^2 = 1$. Set up an integral to find the volume of $B$. \textbf{You do not need to evaluate it.} (Hint: Use a rotated version of spherical coordinates.) 

\vfill

\item (7 pts) Let $S$ denote the sphere of radius $2$ centered at $(0,0,0)$, and suppose $S$ is filled with a fluid with density function $f(x,y,z) = z^3 - z + 8$. Find the total mass of fluid inside $S$ by integrating the function $f$ over $S$. (Hint: use symmetry.)

\vfill
\end{enumerate}


\newpage


\item (12 pts) Use cylindrical coordinates to evaluate 

\[
\iiint_E e^z \, dV,
\]

where $E$ is the region bounded by the parabaloid $z = 4+x^2 + y^2,$ the cylinder $x^2+y^2 =2$, and the plane $z=0$.





%
%
%\newpage
%
%\item (10 pts) Let $D$ be the region in the $x$-$y$ plane bounded by the curves $y = \frac{x}{2}$ and $y^2 - 2y = x$. 
%
%\begin{enumerate} \item Draw a picture of $D$. 
%
%\vspace{4cm} 
%
%\item Write down a parameterization of $D$ in cartesian coordinates. 
%
%\vspace{3cm}
%
%\item Set up and evaluate the integral 
%
%\[
%\iint_D (y+1) \, dA
%\]
%
%using your parameterization from part $b$. 
%
%\end{enumerate}
%
%\vfill

\newpage

\item \begin{enumerate} \item (6 pts) Set up a double integral in polar coordinates to find the area of the region inside the circle $(x-3)^2 + y^2 = 9$ and outside the circle $x^2+y^2 = 9$. \textbf{You do not need to evaluate it.}

\vfill

\item (6 pts) Consider the transformation $T: \mathbb{R}^2 \to \mathbb{R}^2$ given by $(u,v) = T(x,y) = (2x + 4y, x - 3y)$. Solve for the inverse of $T$ in terms of equations $x = x(u,v)$ and $y = y(u,v)$, and find the Jacobian determinant of $T$. 

\vfill 

\end{enumerate}


\newpage


\item (12 pts) Consider the tetrahedron $E \subset \mathbb{R}^3$ bounded by the planes $x = 0, z = 0, z = 2y$ and $2x + 2y + z = 4$. Set up the triple integral 

\[
\iiint_E xz \, dV
\]

with the two given orders of integration. \textbf{You do not need to evaluate the integrals.}

\begin{enumerate} \item $dx \,dy \,dz.$
\vfill


\item $dy \, dz \, dx.$
\vfill
\end{enumerate}


%\item (12 pts) Consider the region in the $x$-$y$ plane inside the circle $x^2 + (y-1)^2 = 9$ and above the line $y = x \sqrt{3}$. 
%
%\begin{enumerate} \item Re-write the circle equation using polar coordinates. 
%
%\vspace{3cm}
%
%\item Solve the equation you got in part $a$ for $r$ as a function of $\theta$. 
%
%\vspace{5cm} 
%
%\item Draw a picture of $D$, and set up an integral in polar to find the area of $D$ using your answer from part $b$. \textbf{You do not need to evaluate it.} 
%
%
%\vfill 

\end{enumerate}

\end{document}
