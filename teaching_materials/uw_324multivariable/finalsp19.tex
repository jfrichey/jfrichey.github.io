\documentclass[12 pt]{report}

\usepackage{epsfig}
\usepackage[in,empty]{myfullpage}
\usepackage{amssymb}
\usepackage{amsmath}

\baselineskip=20pt

\begin{document}

\noindent \vfill \noindent \large

\centerline{Math 324 C  - Spring 2019}

\centerline{Final exam}

\centerline{Monday, June 10, 2019}

\normalsize

\vfill
\medskip
Name: \rule{10cm}{1pt}

\bigskip

\vfill
\begin{center}
{\large
\begin{tabular}{||c|c|r||}
\hline Problem 1 & 4 & \hspace{10mm} \hfill \\
\hline Problem 2 & 5  & \hspace{10mm} \hfill \\
\hline Problem 3 & 5 & \hspace{10mm} \hfill \\
\hline Problem 4 & 5  & \hspace{10mm} \hfill \\
\hline Problem 5 & 3  & \hspace{10mm} \hfill \\
\hline Problem 6 & 3 & \hspace{10mm} \hfill \\
\hline Total & 25 & \hspace{10mm} \hfill \\
\hline
\end{tabular}
}
\end{center}
\vfill
\begin{itemize}
\item There are 5 questions on this exam. Make sure you have all five.
\item \textbf{You must show your work on all problems.}  The correct answer
with no supporting work may result in no credit. Put a box
around your FINAL ANSWER for each problem and cross out any work
that you don't want to be graded.
\item Give exact answers, and simplify as much as possible. 
\item Use the backs of pages \textit{for scratch work only}.
\item Raise your hand if you have a question.
\item Any student found engaging in academic misconduct will receive
a score of 0 on this exam.
\item You have 110 minutes to complete the exam.  Budget your time wisely! \\
\end{itemize}
\vfill
\begin{center}GOOD LUCK!\end{center}

\newpage
\begin{enumerate}

\item Let $S$ be the boundary surface of the cone region $x^2+y^2 \leq z^2 \leq 1$, so $S$ consists of the cone $x^2+y^2 = z^2$ for $0 \leq z \leq 1$ and the disk $x^2+y^2 \leq 1$ in the plane $z=1$. Equip $S$ with the outward pointing normal (normal points away from the cone). Use the divergence theorem to evaluate the surface integral

$$\iint_S \langle 5, xy^2z, -xyz^2 \rangle \cdot d \bold{S}.$$
\newpage

\item[2a.] Show that $\nabla \times \langle 3y, -2yz, \log z \rangle = \langle 2y, 0, -3 \rangle$.

\vspace{5cm} 

\item[2b.] Let $S$ be the part of the ellipsoid $x^2+\frac{y^2}{9}+\frac{z^2}{4} = 5$ lying above the plane $z = 2$ oriented downward. Use Stokes' theorem and the first part of this problem to evaluate 

$$\iint_S \langle 2y, 0, -3 \rangle \cdot d\bold{S}.$$
 
\newpage

\item[3.] Let $F = \langle xy^2, x+y \rangle$ be a vector field in the $xy$-plane, and let $C$ denote the upper half of the unit circle $x^2+y^2 = 1, y \geq 0$ oriented counter-clockwise. Also, let $D$ be the upper half of the unit disk $x^2 + y^2 \leq 1, y \geq 0$. 
\begin{enumerate}
\item[a.] Draw a picture of $D$ and $C$, and evaluate $\int_C F \cdot d\bold{r}$. 

\vspace{7cm}

\item[b.] Henry, Jacob's evil twin brother, claims he is able to use Green's theorem to solve part $a$. He reasons as follows: ``The curve $C$ bounds the region $D$ with the positive orientation, so by Green's theorem, 

$$\int_C F \cdot d\bold{r} = \iint_D \Big(\frac{d}{dx}(x+y) - \frac{d}{dy}(xy^2) \Big) \, dA = \int_0^\pi \int_0^1 (1-2 r^2 \sin\theta \cos\theta) r dr d\theta = \frac{\pi}{2}."$$

What is wrong with Henry's argument?   

\vspace{3cm} 

\item[c.] Use Green's theorem correctly to relate a double integral over $D$ to a line integral. Explain why Henry got the right answer, even though his reasoning is flawed. 



\end{enumerate} 

\newpage

\item[4a.] Show that $\nabla \cdot (\nabla \times F) = 0$ for any smooth vector field $F = \langle P, Q, R \rangle$ on $\mathbb{R}^3$.  

\vspace{10cm}

\item[4b.] Use the result from part $a$ to show that 

$$\iint_S \text{curl}(F) \cdot d\bold{S} = 0$$ 

where $S$ is the surface $x^2 + (y-1)^2 + (z+1)^2 = 3$, and $F = \langle 9y^2 x^3+3z^4, 2xz^2, z^5 \rangle$. [If you don't know how to use the result from part a, but have some other way of showing this fact, you can receive partial credit on this problem.]

\newpage

\item[5.] For each of the statements below, circle \textbf{True} if you think it is true, and circle \textbf{False} if you think it is false. You may use the space to do scratch work, but no partial credit will be awarded on this question. 

\begin{enumerate} 
\item \textbf{True} \hspace{5pt} \textbf{False} \hspace{10pt} The vector field $G(x,y,z) = \langle 3x - 2y e^x, ye^y+z, 3x \rangle$ satisfies 

\[
\nabla \cdot (\nabla \times (\nabla \times \nabla(\nabla \cdot G))) = -2ye^x+e^y.
\]

\vfill

\item \textbf{True} \hspace{5pt} \textbf{False} \hspace{10pt} Any surface $S \subset \mathbb{R}^3$ can be given an orientation (i.e. continuous choice of normal vector $\hat{n}$) so that the integral 

\[
\iint_S F \cdot d\bold{S} = \iint_S F \cdot \hat{n} \, dS
\]

exists.

\vfill

\item \textbf{True} \hspace{5pt} \textbf{False} \hspace{10pt} There exists a function $h(x,y,z)$ and a closed loop $C$ in $\mathbb{R}^3$ (same starting and ending point) so that 

\[
\int_C h \cdot ds \neq 0.
\]

\vfill

\end{enumerate}

\newpage

\item[6.] Let $C$ be the cylinder $x^2 + y^2 \leq 1$ for $0 \leq z \leq 1$, and let $S$ be the boundary surface of $C$, oriented outward, so $S$ consists of the cylinder for $0 < z < 1$ and the disks of radius 1 in the planes $z = 0$ and $z = 1$. Without using the divergence theorem or evaluating any integrals, explain the following two (true) equalities:

\[
\iint_S z\hat{k} \cdot d\bold{S} = \pi,
\]

and

\[
\iint_S |x| \hat{i} \cdot d\bold{S} = 0.
\]

\vfill

\end{enumerate}

\end{document}
