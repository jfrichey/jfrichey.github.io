\documentclass[11 pt]{report}

\usepackage{myfullpage}
\usepackage{amssymb}
\usepackage{esint}

\pagestyle{empty}

\baselineskip=20pt

\newcommand{\pp}{\par \noindent}
\begin{document}

\centerline{\bf Math 324, Homework 1}
%
%\begin{center} Due Friday, March 31, at the start of class. \end{center}

\vspace{.2cm}

\noindent \textbf{Stewart 15.2} For 7, 9, 13, and 22, calculate the value of the double integral. 
 
\vspace{.2cm}

$\# 7.$ $\displaystyle{\int_{-3}^3 \int_0^{\pi/2} (y+y^2 \cos x) dx dy.}$

\vspace{.2cm}

$\# 9.$ $\displaystyle{\int_1^4 \int_1^2 (\frac{y}{x}+ \frac{x}{y}) dy dx.}$

\vspace{.2cm}

$\# 13.$ $\displaystyle{\int_0^2 \int_0^\pi r \sin^2 \theta d\theta dr.}$

\vspace{.2cm}

$\# 22.$ $\displaystyle{\iint_R \frac{1}{1+x+y} \,dA, R = [1,3] \times [1,2].}$

\vspace{.2cm}

$\# 29.$ Find the volume of the solid enclosed by the surface 

$$z = x \sec^2 y$$ 

and the planes 

$$z = 0, x = 0, x = 2, y = 0, y = \pi / 4.$$

\vspace{.3cm} 

\noindent \textbf{Stewart 15.3} For 51 and 53, evaluate the double integral by reversing the order of integration. 

\vspace{.2cm}

$\# 6.$ Evaluate the integral. $\displaystyle{\int_0^1 \int_0^{e^v} \sqrt{1+e^v} dw dv.}$

\vspace{.2cm} 

$\# 19.$ $\displaystyle{\iint_D y^2 dA,}$ where $D$ is the triangular region with vertices $(0,1), (1,2), (4,1)$. 

\vspace{.2cm}

$\# 51.$ $\displaystyle{\int_0^4 \int_{\sqrt{x}}^2 \frac{1}{y^3+1} dy dx.}$

\vspace{.2cm}

$\# 53.$ $\displaystyle{\int_0^1 \int_{\arcsin(y)}^{\pi/2} \cos (x) \sqrt{1+\cos^2 (x)} dx dy}.$

\vspace{.4cm}

\noindent \textbf{Stewart 15.4} 

\vspace{.2cm}

$\# 5.$ Sketch the region whose area is given by the integral and evaluate the integral. $\displaystyle{\int_{\pi/4}^{3\pi/4} \int_1^2 r dr d\theta.}$

\vspace{.2cm}

$\# 11.$ Evaluate the integral by writing it in polar coordinates. $\displaystyle{\iint_D e^{-x^2-y^2} dA,}$ where $D$ is the region bounded 

by the semicircle $x = \sqrt{4-y^2}$ and the $y-$axis. 

\vspace{.2cm}

$\# 18.$ Use a double integral to find the area of the region inside the cardioid $r = 1+\cos\theta$ and outside the 

circle $r = 3 \cos \theta.$ 

\vspace{.2cm}

$\# 26.$ Use polar coordinates to find the volume of the solid bounded by the paraboloids $z = 3x^2+3y^2$ and 

$z = 4-x^2-y^2$. 

\vspace{.3cm}

\end{document}