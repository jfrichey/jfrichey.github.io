\documentclass[12 pt]{report}

\usepackage{epsfig}
\usepackage[in,empty]{myfullpage}
\usepackage{amssymb}
\usepackage{amsmath}

\baselineskip=20pt

\begin{document}

\noindent \vfill \noindent \large

\centerline{Math 324 B  - Autumn 2018}

\centerline{Final exam}

\centerline{Wednesday, December 12th, 2018}

\normalsize

\vfill
\medskip
Name: \rule{10cm}{1pt}

\bigskip

\vfill
\begin{center}
{\large
\begin{tabular}{||c|c|r||}
\hline Problem 1 & 15 & \hspace{10mm} \hfill \\
\hline Problem 2 & 15 & \hspace{10mm} \hfill \\
\hline Problem 3 & 19 & \hspace{10mm} \hfill \\
\hline Problem 4 & 16  & \hspace{10mm} \hfill \\
\hline Problem 5 & 20  & \hspace{10mm} \hfill \\
\hline Problem 6 &  15 & \hspace{10mm} \hfill \\
\hline Total & 100 & \hspace{10mm} \hfill \\
\hline
\end{tabular}
}
\end{center}
\vfill
\begin{itemize}
\item There are 6 problems (8 pages) in this exam. Make sure you have them all. 
\item You must show your work on all problems.  The correct answer
with no supporting work may result in no credit. \textbf{Put a box
around your FINAL ANSWER for each problem and cross out any work
that you don't want to be graded.} 
\item Give exact answers, and simplify as much as possible. 
For example, $\frac{\pi}{\sqrt{2}}$ is acceptable, but $\frac{1}{2} + \frac{3}{4}$
should be simplified to $\frac{5}{4}$.   

\item If you need more room, use the backs
of the pages and indicate to the grader that you have done so.
\item Raise your hand if you have a question.
\item Any student found engaging in academic misconduct will receive
a score of 0 on this exam.
\item You have 2 hours to complete the exam.  Budget your time wisely! \\
\end{itemize}
\vfill
\begin{center}GOOD LUCK!\end{center}

\newpage
\begin{enumerate}

\item (15 pts) Let $C$ be the cylinder $x^2+z^2 = 4$ for $-1 \leq y \leq 6$, oriented inward (i.e. normal points toward the $y$-axis). Use Stokes' theorem to evaluate

\[
\iint_C \nabla \times \langle yz^2, 0, 0 \rangle \, \cdot dS.
\]

Make sure to indicate how you are orienting the boundary. 

\vfill

\newpage \item (15 pts) Evaluate the surface integral

\[
\iint_S \langle x+y, z, z-x \rangle \, \cdot dS,
\]

where $S$ is the boundary of the region between the paraboloid $z = 9-x^2-y^2$ and the $xy$-plane, oriented inward.

\vfill

\newpage \item (19 pts) Consider a uniform magnetic field $B$ with constant strength $b > 0$ in the $z$-direction, i.e. $B$ is the vector field $B = b \hat{k}$.

\begin{enumerate} \item (5 pts) Let $r$ be the vector field $r = x \hat{i} + y \hat{j}$. Verify that $A = \frac{1}{2} B \times r$ is a `vector potential' for $B$, i.e. $\nabla \times A = B$. 

\vfill

\item (6 pts) Calculate the flux of $B$ through the disk $x^2 + y^2 \leq 1$ in the plane $z = 1$, oriented upward. 

\vspace{4cm}

\item (8 pts) Use the result from part $(a)$ and Stokes' theorem to calculate the flux of $B$ through the disk bounded by the curve $s(t) = \cos t \hat{i} + \frac{\sqrt{2}}{2} \sin t \hat{j} - \frac{\sqrt{2}}{2} \sin t \hat{k}$ for $0 \leq t \leq 2\pi$. 

\vfill

\end{enumerate}

\newpage \item (16 pts; 4 pts each) For each of the statements below, circle \textbf{True} if you think it is true and \textbf{False} if you think it is false. 

\begin{enumerate} \item \textbf{True}\hspace{5pt} \textbf{False} \hspace{5pt} The electric field $E$ given by 

\[
E = \frac{\epsilon Q}{\rho^3} \vec{\rho}
\]

satisfies $\nabla \cdot E = 0$. ($\vec{\rho} = \langle x,y,z \rangle,$ and $\rho = ||\vec{\rho}\,||$.) 

\vfill

\item \textbf{True} \hspace{5pt} \textbf{False} \hspace{5pt} Suppose $F$ and $G$ are vector fields with $\nabla \times F = G$. Then there exists a vector field $H$ which is different from $F$ and satisfies $\nabla \times H = G$. 

\vfill

\item \textbf{True} \hspace{5pt} \textbf{False} \hspace{5pt} If $G$ is a vector field and $\nabla \cdot G = 0$, then 

\[
\iint_S \nabla \times G \cdot dS = 0
\]

for any oriented surface $S$. 

\vfill

\item \textbf{True} \hspace{5pt} \textbf{False} \hspace{5pt} There exists a vector field $F$ such that 

\[
\nabla \Big(\nabla \cdot \Big(\nabla \times \nabla\Big( \nabla \cdot F\Big)\Big)\Big)
\]

is not the 0 vector field. 

\vfill

\end{enumerate}

%\newpage
%
%\item (20 pts) Let $E$ denote the unit cube in $\mathbb{R}^3$, $E = \{(x,y,z): 0 \leq x, y, z \leq 1\}$. Let $S = \partial E$ be the boundary of the cube, with the outward pointing orientation (away from $E$). Use the divergence theorem to evaluate 
%
%\[
%\iint_{\partial E} \langle 0, 0, e^{y+z} \rangle \cdot dS.
%\]

\newpage

 \item (20 pts) Consider the radial vector field $F = \frac{1}{\rho^4} \vec{\rho}$, where $\rho$ and $\vec{\rho}$ are defined as in problem 4(a).

\begin{enumerate} \item (5 pts) For vector fields of the form $G = g(\rho) \vec{\rho}$, the divergence can be written as 

\[
\nabla \cdot G = \frac{1}{\rho^2} \frac{\partial }{\partial \rho} \Big( \rho^3 g(\rho) \Big)
\]

Use this formula to find the divergence of $F$ for $\rho > 0$. Write your final answer in terms of $\rho$.

\vfill

\item (7 pts) Let $E_R$ be the spherical region $\{(\rho, \theta, \phi): 1 \leq \rho \leq R\}$, where $R > 1$ is a fixed number. Evaluate the volume integral

\[
\iiint_{E_R} \nabla \cdot F \, dV.
\]

\vfill

\newpage

\item (5 pts) Let $S_1$ denote the surface of the sphere of radius $1$ oriented toward the origin, and let $S_R$ denote the surface of the sphere of radius $R$ oriented away from the origin. According to the divergence theorem, 

\[
\iint_{S_1} F \cdot dS + \iint_{S_R} F \cdot dS = \iiint_{E_R} \nabla \cdot F \, \cdot dS.
\]

You found the value of the right hand side of this equation in part b: compute one of the integrals on the left hand side, and use your answer to find the value of the other integral. 

\vfill 

\item (3 pts) What happens to the values of the three integrals from part $c$ when $R \to \infty$?

\vfill

\end{enumerate}


\newpage

\item (15 pts) Let $F = \langle xy^2, x+y \rangle$ be a vector field in the $xy-$plane, and let $C$ denote the upper half of the unit circle $x^2+y^2 = 1, y \geq 0$ oriented counter-clockwise. Also, let $D$ be the upper half of the unit disk $x^2 + y^2 \leq 1, y \geq 0$. 
\begin{enumerate}
\item (6 pts) Draw a picture of $D$ and $C$, and evaluate $\int_C F \cdot d\bold{r}$. 

\vspace{7cm}

\item (4 pts) Henry, Jacob's evil twin brother, claims he is able to use Green's theorem to solve part $a$. He reasons as follows: ``The curve $C$ bounds the region $D$ with the positive orientation, so by Green's theorem, 

$$\int_C F \cdot d\bold{r} = \iint_D \Big(\frac{d}{dx}(x+y) - \frac{d}{dy}(xy^2) \Big) \, dA = \int_0^\pi \int_0^1 (1-2r^2 \sin\theta \cos\theta) r dr d\theta = \frac{\pi}{2}."$$

What is wrong with Henry's argument?   

\vspace{3cm} 

\item (5 pts) Use Green's theorem correctly to relate a double integral over $D$ to a line integral. Explain why Henry got the right answer, even though his reasoning is flawed. 



\end{enumerate} 


\end{enumerate}


\end{document}
