\documentclass[12 pt]{report}

\usepackage{epsfig}
\usepackage[in,empty]{myfullpage}
\usepackage{amssymb}
\usepackage{amsmath}

\baselineskip=20pt

\begin{document}

\noindent \vfill \noindent \large

\centerline{Math 324 E  - Fall 2017}

\centerline{Midterm exam 2}

\centerline{Wednesday, November 8, 2017}

\normalsize

\vfill
\medskip
Name: \rule{10cm}{1pt}

\bigskip

\vfill
\begin{center}
{\large
\begin{tabular}{||c|c|r||}
\hline Problem 1 & 10 & \hspace{10mm} \hfill \\
\hline Problem 2 & 14 & \hspace{10mm} \hfill \\
\hline Problem 3 & 16 & \hspace{10mm} \hfill \\
\hline Problem 4 & 10  & \hspace{10mm} \hfill \\
\hline Total & 50 & \hspace{10mm} \hfill \\
\hline
\end{tabular}
}
\end{center}
\vfill
\begin{itemize}
\item There are 4 problems on this exam. Make sure you have all four.
\item You must show your work on all problems.  The correct answer
with no supporting work may result in no credit. \textbf{Put a box
around your FINAL ANSWER for each problem and cross out any work
that you don't want to be graded.} 
\item Give exact answers, and simplify as much as possible. 
For example, $\frac{\pi}{\sqrt{2}}$ is acceptable, but $\frac{1}{2} + \frac{3}{4}$
should be simplified to $\frac{5}{4}$.   

\item If you need more room, use the backs
of the pages and indicate to the grader that you have done so.
\item Raise your hand if you have a question.
\item Any student found engaging in academic misconduct will receive
a score of 0 on this exam.
\item You have 50 minutes to complete the exam.  Budget your time wisely! \\
\end{itemize}
\vfill
\begin{center}GOOD LUCK!\end{center}

\newpage
\begin{enumerate}

\item (10 points) Let $f(x,y,z) = xz^2 - 2yz$. 

\begin{enumerate} \item[a)] Compute the directional derivative of $f$ at the point $(-1,0,2)$ in the direction $u = \frac{1}{\sqrt{3}} (\hat{i} + \hat{j} + \hat{k})$. 

\vfill 

\item[b)] What is the maximum value of directional derivative at $(-1,0,2)$, and which direction does it occur in?

\vfill

\end{enumerate}

\newpage

\item (14 points) For each pair of conservative vector field $F$ and curve $C$, first find a potential function for $F$, and then use the fundamental theorem of line integrals to evaluate $\int_C F \cdot dr$. 

\begin{enumerate} \item[a)] $F(x,y) = xy \hat{i} + \frac{1}{2}x^2 \hat{j}$, and $C$ is the part of the hyperbola $y = \frac{3}{x}$ between the points $(1,3)$ and $(3,1)$, traversed from left to right. 

\vfill

\item[b)] $F(x,y) = x^{-2} y^{-1} \hat{i} + x^{-1}y^{-2} \hat{j},$ and $C$ is the infinite(!) ray along the line $x = 2y$ for $x \geq 1$, with initial point $(1,2)$. (Hint: do the problem with the part of the ray out to the point $(n, 2n)$, and then let $n \to \infty.$)

\vfill

\end{enumerate}

\newpage


\item (16 points) For each of the statements below, circle \textbf{True} if you think it is true, and circle \textbf{False} if you think it is false. You may use the space to do scratch work, but no partial credit will be awarded on this question. Note that for a vector field $F = P \hat{i} + Q \hat{j}$ and a function $g$, we define $gF$ as the vector field $gP \hat{i} + gQ \hat{j}$. (4 points for each statement.)

\begin{enumerate} 
\item \textbf{True} \hspace{5pt} \textbf{False} \hspace{10pt} The function $f(x,y) = e^{x^2-2y^2}$ satisfies $\nabla f = f(2x \hat{i} - 4y \hat{j})$.

\vfill

\item \textbf{True} \hspace{5pt} \textbf{False} \hspace{10pt} The vector field $F = \frac{-y \hat{i} + x \hat{j}}{x^2+y^2}$ satisfies $\int_C F \cdot dr = 0$ where $C$ is the unit circle traversed counter clockwise. 

\vfill

\item \textbf{True} \hspace{5pt} \textbf{False} \hspace{10pt} Let $G(x,y) = 6y \cos(2x) \hat{i} + 6 \sin(x) \cos(x) \hat{j}$. $G$ is a conservative vector field. 

\vfill

\item \textbf{True} \hspace{5pt} \textbf{False} \hspace{10pt} Let $C$ denote the part of the parabola $y = 1-x^2$ between the points $(0,1)$ and $(4,-15)$, traversed from right to left. Then the vector $r(t) = 2t \hat{i}-4t^3 \hat{j}$ is tangent to $C$ for $0 \leq t \leq 2$. 

\vfill

\end{enumerate}

\newpage


\item (10 points) Let $F(x,y) = 2x^2y \hat{i} - 3x \hat{j}$, and let $C$ denote the curve defined by the ellipse $x^2 + \frac{y^2}{9} = 1$, traversed clockwise. Use Green's theorem to evaluate $\int_C F \cdot dr$. 

\end{enumerate}


\end{document}
