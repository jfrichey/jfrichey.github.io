\documentclass[11 pt]{report}

\usepackage{myfullpage}
\usepackage{amssymb}

\pagestyle{empty}

\baselineskip=20pt

\newcommand{\pp}{\par \noindent}
\begin{document}

\centerline{\bf Advanced Multivariable Calculus I: Math 324 B - Autumn 2018}
\vspace{0.2cm}
\begin{tabular}{lllll}
{\bf Lecturer: }    & Jacob Richey &  & {\bf Email: }       & jfrichey@math.washington.edu \\
{\bf Office: }      & Padelford C-8K        &  & {\bf Web page: }    & www.math.washington.edu/$\sim$jfrichey \\
\end{tabular}
\vspace{0.25cm}

\noindent {\bf Office Hours: W 11-12, Th 2-3.}   
I will always be in my office during office hours. If you can't make it, let me know and we can set up another time to meet.  \vspace{0.25cm}

\noindent {\bf Text:} \emph{Calculus: Early Transcendentals}, by James Stewart, 7th
Edition. Note: We're using a custom edition of Stewart's Calculus,
available at the University Bookstore.  There are two volumes:
Volume 1 covers Math 124/125, Volume 2 covers Math 126/324. 

\vspace{0.25cm}

\noindent {\bf Course Objectives:}  This course is a continuation of
Math 126.  The focus is mostly on integration in multiple variables.
We discuss Chapter 15: iterated integrals (double and triple), a bit
of Chapter 14: Gradient and Derivatives, and then the rest of the
term is about Chapter 16.  Chapter 16 introduces line integrals,
vector fields, surface integrals, and ultimately how to calculate
them using the theorems of Green, Stokes, and Gauss.  This course is
`end-loaded', in that there are a lot of big topics in the last few
weeks.  So be ready for that!

\vspace{0.25cm}

\noindent {\bf Grading:} The weight for each part of the course is given
below. \textbf{Midterm dates are tentative.}

\vspace{-.25in}

\begin{center}\begin{tabular}{lccl} &  \\
\underline{Category} & \underline{Weight}  \\
Homework   ({\bf via Webassign})                 & 5  \\
Participation and Quizzes  & 15 \\
Midterm 1  ({\bf Monday, October 22})                   & 25  \\
Midterm 2  ({\bf Monday, November 19})                   & 25   \\
Final Exam ({\bf Wednesday, December 12, 8:30-10:20 in CDH 110})                        & 30  \\
\hline Total                                        & 100
\end{tabular}
\end{center}

\vspace{0.25cm}

\noindent {\bf Lecture:}  Lecture is on Monday, Wednesday, and Friday, 9:30-10:20 AM in CDH 110.  You
are responsible for all information that is discussed
during lecture. Usually, Mondays and Wednesdays will be lectures, and we will devote Fridays to solving problems in class. 
\vspace{0.25cm}

\noindent {\bf Participation and Quizzes:}  I will post reading questions online after each class which pertain to the material we will cover in the next class. I expect you to read and think about these questions \emph{before} you come to class. \emph{To test your preperation, I will call on one or two students randomly at the beginning of each class to summarize the material covered in the text.} There will be short (5 minute) quizzes on most Fridays, which will consist of one or two easy problems. 

\vspace{0.25cm}

\noindent {\bf Homework:}  Homework will be assigned weekly through webassign, and will (generally) be due each Wednesday. You may need a scientific calculator for solving homework problems in Math 324.  It must have trigonometric functions, like Sin and Cos, as well as logarithms and exponentials (ln and exp).
\vspace{0.25cm}


\noindent {\bf Exams:}  The midterms will be 50 minutes long and
will be given at lecture.  The final exam is cumulative. The date of the final is set by the university and is very unlikely to change under any circumstances, so you should plan your travel arrangements accordingly. During exams you are allowed one 
sheet of hand-written notes (8.5x11 inches, double sided), and a TI 30X IIS calculator: no other calculators are permitted. 
Cheating will not be tolerated. 

\vspace{.25cm}

\noindent {\bf Make-Ups:}  \emph{Late homework will not be accepted
for any reason.}  In case of observance of religious holidays or
participation in university sponsored activities, arrangements must
be made at least 1 month in advance for exams. You will be required
to provide documentation for your absence. \emph{Make-up exams will
not be given}.  If you miss an exam due to {\bf unavoidable,
compelling, and well-documented} circumstances, the other exams will
be weighted more heavily. \vspace{0.25cm}

\newpage 

\noindent {\bf Tips for succeeding in this class:}. \\
{\bf1.  Homework is crucial:} Mathematics is not a spectator sport: to learn it, you have to do it.
Reading the text and paying attention in lecture are just as important as thinking about
the material on your own. When you are stuck or confused on a problem, don't immediately check your notes or 
ask a friend to find the solution: being stuck is where the most valuable learning can occur! 

\vspace{.2cm}

\noindent Try to adopt good work habits: look at
the homework within the first few days it is assigned, so your mind can have time to ruminate on the conecptually difficult problems.
Absorbing mathematical ideas is like eating: it is better to have one meal of math each day, rather than five in one day, so you can digest properly. If you cram too much math the night before an exam, you are bound to puke it all up the next morning.  

\vspace{.25cm}  

\noindent {\bf2.  Ask for help:} Most students will hit a wall at some point
during the course.  Some can't handle the large workload, while
others find difficulty with specific concepts in the course. When
these times arrive remember to ask for help.  Come to me, ask your classmates for help, visit CLUE and/or visit the student counseling center.  These are just a
few of your options. It is better to find help earlier rather
than later. You are all smart enough to do well in this course: the
question is whether or not you are determined enough.
\vspace{0.3cm}

\noindent {\bf Resources:} \\
\noindent $\bullet$\ A link to the class website can be
found at: {\bf http://www.math.washington.edu/$\sim$jfrichey/} \\ You
will find homework assignments, review sheets, grade information, a
calendar for the term, and various bits of other useful information
there, including past exams and quizzes, TA information, etc.


\vspace{.2cm}

\noindent $\bullet$\ The Center for Learning and Undergraduate
Enrichment (CLUE) holds drop-in tutoring sessions every weekday
evening in Mary Gates Hall Commons.  See {\bf http://depts.washington.edu/clue/} for more details.

\vspace{.2cm}

\noindent $\bullet$\ The University of Washington is committed to providing
access, equal opportunity and reasonable accommodation in its
services, programs, activities, education and employment for
individuals with disabilities.  To request disability accommodation
contact the Disability Services Office at least ten days in advance
at: 206-543-6450 (V), 206-543-6452 (TTY), 206-685-7264(FAX), or
dso@u.washington.edu.

\vspace{.2cm}

\noindent $\bullet$\  The Student Counseling Center provides academic skills
workshop on a variety of topics including stress management test
anxiety and time management to help you succeed at the University of
Washington. If any of these is an issue for you, check out the
schedule of workshops at {\bf
http://depts.washington.edu/scc/studyskills.html} .

\end{document}
