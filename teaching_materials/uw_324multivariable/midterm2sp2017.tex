\documentclass[12 pt]{report}

\usepackage{epsfig}
\usepackage[in,empty]{myfullpage}
\usepackage{amssymb}
\usepackage{amsmath}
\usepackage{tikz}

\unitlength = .5cm 

\baselineskip=20pt

\begin{document}

\noindent \vfill \noindent \large

\centerline{Math 324 A  - Spring 2017}

\centerline{Midterm exam 2}

\centerline{Friday, May 12th, 2017}

\normalsize

\vfill
\medskip
Name: \rule{10cm}{1pt}

\bigskip

\vfill
\begin{center}
{\large
\begin{tabular}{||c|c|r||}
\hline Problem 1 & 12 & \hspace{10mm} \hfill \\
\hline Problem 2 & 12  & \hspace{10mm} \hfill \\
\hline Problem 3 & 14 & \hspace{10mm} \hfill \\
\hline Problem 4 & 12  & \hspace{10mm} \hfill \\
\hline Total & 50 & \hspace{10mm} \hfill \\
\hline
\end{tabular}
}
\end{center}
\vfill
\begin{itemize}
\item There are 4 questions on this exam. Make sure you have all four.
\item You must show your work on all problems.  The correct answer
with no supporting work may result in no credit. \textbf{Put a box
around your FINAL ANSWER for each problem and cross out any work
that you don't want to be graded.} 
\item Give exact answers, and simplify as much as possible. 
For example, $\frac{\pi}{\sqrt{2}}$ is acceptable, but $3/4 + 1/2$
should be reduced to $5/4$.   

\item If you need more room, use the backs
of the pages and indicate to the grader that you have done so.
\item Raise your hand if you have a question.
\item Any student found engaging in academic misconduct will receive
a score of 0 on this exam.
\item You have 50 minutes to complete the exam.  Budget your time wisely! \\
\end{itemize}
\vfill
\begin{center}GOOD LUCK!\end{center}

\newpage
\begin{enumerate}

\item (12 pts) Consider the function $f(x,y,z) = \frac{1}{3}x^2 \sin(y) - 2xz.$ 


\begin{enumerate} \item[a.] Find $\frac{\partial f}{\partial x}, \frac{\partial f}{\partial y}$ and $\frac{\partial f}{\partial z}$.

\item[b.] Let $v$ be the unit vector with tail at the origin in $\mathbb{R}^3$ and tip at the point $(1, \frac{\pi}{3}, \frac{\pi}{2})$ in spherical coordinates $(\rho, \theta, \phi)$. Find the value of the directional derivative $D_vf(x,y,z)$ at the point $(x,y,z) = (3, -\pi ,1)$. 

\item[c.] Give the unit vector $u$ that maximizes $D_uf(3,-\pi,1)$.  

\item[d.] Let $F$ be the gradient vector field of $f$, so $F = \nabla f$. Find the curl and divergence of $F$.

\end{enumerate} 

\newpage

\item[2a.] (4 pts) Let $f(x,y) = \frac{1}{2}(x^2 + y^2)$. Write down a formula for $\nabla f(x,y)$, and draw the vector $\nabla f(x,y)$ at each of the three points $(1,0), (-2,1)$ and $(2,-2)$.  

\vspace{1cm}

\begin{picture}(10.5,10.5)(-5,-5)
  {\color{gray}
  \thinlines
  \multiput(-5,-4)(0,1){9}{\line(1,0){10}}
  \multiput(-4,-5)(1,0){9}{\line(0,1){10}}
  }
  \thicklines
  \put(-5,0){\vector(1,0){10.2}}
  \put(0,-5){\vector(0,1){10.2}}
  \put(5.3,0){\makebox(1,0)[l]{$x$}}
  \put(0,5.3){\makebox(0,1)[b]{$y$}}
\end{picture}

\vspace{1cm}

\item[2b.] (4 pts) Define what it means for a vector field $F$ defined on an \emph{arbitrary domain} $D$ in the plane to be conservative. (You may give any definition equivalent to the one we used in class.)

\vspace{4cm}

\item[2c.] (4 pts) Consider the vector field $F(x,y) = \langle y^3 \cos(x), - 3y^2 \sin(x) \rangle$. Is $F$ conservative? If so, give a potential function; if not, explain how you know it isn't conservative. 

\newpage

\item[3a.] (7 pts) State the fundamental theorem for line integrals, and use it to compute 

$$\int_C \sin(y) e^{x \sin(y)} dx + x \cos(y) e^{x \sin(y)} dy,$$

where $C$ is the line segment from $(1,\pi)$ to $(2,\pi)$ followed by the line segment from $(2,\pi)$ to $(2,2 \pi)$. 

\vspace{7cm}  

\item[3b.] (7 pts) Let $F$ denote the force field $F(x,y,z) = \langle 3z^2 \ln (2xy-1), x^2+(z+1)y^{-2}, xz-1 \rangle$. Find the work done by $F$ on a particle that travels along the path $r(t) = (t, t^{-1}, 3), 1 \leq t \leq 2$. 


\newpage

\item[4.] (12 pts) Let $C$ denote the triangle with vertices $(-2,-1), (2,-1)$, and $(0,6)$, oriented counter clockwise. Let $G$ be the vector field $G(x,y) = \langle 3xy^2, x+y \rangle.$ Evaluate

\[
\int_C G \cdot \, dr.
\]

[Hint: apply a theorem, then use symmetry to simplify the integral.]

\end{enumerate}

\end{document}
