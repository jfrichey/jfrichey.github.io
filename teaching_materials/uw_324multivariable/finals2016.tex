\documentclass[12 pt]{report}

\usepackage{epsfig}
\usepackage[in,empty]{myfullpage}
\usepackage{amssymb}
\usepackage{amsmath}

\baselineskip=20pt

\begin{document}

\noindent \vfill \noindent \large

\centerline{Math 324 C  - Summer 2016}

\centerline{Final exam}

\centerline{Friday, August 19th, 2016}

\normalsize

\vfill
\medskip
Name: \rule{10cm}{1pt}

\bigskip

\vfill
\begin{center}
{\large
\begin{tabular}{||c|c|r||}
\hline Problem 1 & 10 & \hspace{10mm} \hfill \\
\hline Problem 2 & 10  & \hspace{10mm} \hfill \\
\hline Problem 3 & 10 & \hspace{10mm} \hfill \\
\hline Problem 4 & 10  & \hspace{10mm} \hfill \\
\hline Problem 5 & 10  & \hspace{10mm} \hfill \\
\hline Total & 50 & \hspace{10mm} \hfill \\
\hline
\end{tabular}
}
\end{center}
\vfill
\begin{itemize}
\item There are 5 questions on this exam. Make sure you have all five.
\item You must show your work on all problems.  The correct answer
with no supporting work may result in no credit. \textbf{Put a box
around your FINAL ANSWER for each problem and cross out any work
that you don't want to be graded.} 
\item Give exact answers, and simplify as much as possible. 
For example, $\frac{\pi}{\sqrt{2}}$ is acceptable, but $3\sqrt{3}+\frac{1}{\sqrt{3}}$
should be reduced to $\frac{10\sqrt{3}}{3}$.   

\item If you need more room, use the backs
of the pages and indicate to the grader that you have done so.
\item Raise your hand if you have a question.
\item Any student found engaging in academic misconduct will receive
a score of 0 on this exam.
\item You have 60 minutes to complete the exam.  Budget your time wisely! \\
\end{itemize}
\vfill
\begin{center}GOOD LUCK!\end{center}

\newpage
\begin{enumerate}

\item (10 pts) Let $S$ be the boundary surface of the cone region $x^2+y^2 \leq z^2 \leq 1$, so $S$ consists of the cone $x^2+y^2 = z^2$ for $0 \leq z \leq 1$ and the disk $x^2+y^2 \leq 1$ in the plane $z=1$. Equip $S$ with the positive orientation, i.e. the outward pointing normal. Use the divergence theorem to evaluate the surface integral

$$\iint_S \langle x^3, xy^2z, -xyz^2 \rangle \cdot d \bold{S}.$$
\newpage

\item[2a.] (3 pts) Show that $\nabla \times \langle 3y, -2yz, \log z \rangle = \langle 2y, 0, -3 \rangle$.

\vspace{5cm} 

\item[2b.] (7 pts) Let $S$ be the part of the ellipsoid $x^2+\frac{y^2}{9}+\frac{z^2}{4} = 5$ lying above the plane $z = 2$ oriented downward. Use Stokes' theorem and the first part of this problem to evaluate 

$$\iint_S \langle 2y, 0, -3 \rangle \cdot d\bold{S}.$$
 
\newpage

\item[3.] Let $F = \langle xy^2, x+y \rangle$ be a vector field in the $xy$-plane, and let $C$ denote the upper half of the unit circle $x^2+y^2 = 1, y \geq 0$ oriented counter-clockwise. Also, let $D$ be the upper half of the unit disk $x^2 + y^2 \leq 1, y \geq 0$. 
\begin{enumerate}
\item[a.] (4 pts) Draw a picture of $D$ and $C$, and evaluate $\int_C F \cdot d\bold{r}$. 

\vspace{7cm}

\item[b.] (2 pts) Henry, Jacob's evil twin brother, claims he is able to use Green's theorem to solve part $a$. He reasons as follows: ``The curve $C$ bounds the region $D$ with the positive orientation, so by Green's theorem, 

$$\int_C F \cdot d\bold{r} = \iint_D \Big(\frac{d}{dx}(x+y) - \frac{d}{dy}(xy^2) \Big) \, dA = \int_0^\pi \int_0^1 (1-2 r^2 \sin\theta \cos\theta) r dr d\theta = \frac{\pi}{2}."$$

What is wrong with Henry's argument?   

\vspace{3cm} 

\item[c.] (4 pts) Use Green's theorem correctly to relate a double integral over $D$ to a line integral. Explain why Henry got the right answer, even though his reasoning is flawed. 



\end{enumerate} 

\newpage

\item[4a.] (5 pts) Show that $\nabla \cdot (\nabla \times F) = 0$ for any smooth vector field $F$ on $\mathbb{R}^3$.  

\vspace{10cm}

\item[4b.] (5 pts) Use the result from part $a$ to show that for any oriented surface $S$ enclosing a region $E$,

$$\iint_S \text{curl}(F) \cdot d\bold{S} = 0.$$ 

[If you don't know how to use the result from part a, but have some other way of showing this fact, you can receive partial credit on this problem.]

\newpage

\item[5.] (10 pts) Let $S$ be the surface parameterized by 

$$r(\phi, \theta) = \sin^2 \phi \cos^2 \theta \hat{i} + \frac{1}{2} \sin^2 \phi \sin^2\theta \hat{j}-\cos^2\phi\hat{k}$$

for $0 \leq \phi \leq \frac{\pi}{2}, 0 \leq \theta \leq \frac{\pi}{2}$. Describe the surface $S$, and find the surface area of $S$. [Hint: try to write down a linear equation $ax+by+cz = d$ relating the $x,y$ and $z$ components of the parameterization.]
\end{enumerate}

\end{document}
