\documentclass[12 pt]{report}

\usepackage{epsfig}
\usepackage[in,empty]{myfullpage}
\usepackage{amssymb}
\usepackage{amsmath}

\baselineskip=20pt

\begin{document}

\noindent \vfill \noindent \large

\centerline{Math 324 E}

\centerline{Midterm exam 2}

\centerline{Monday, November 18th, 2019}

\normalsize

\vfill
\medskip
Name: \rule{10cm}{1pt}

\bigskip

\vfill
\begin{center}
{\large
\begin{tabular}{||c|c|r||}
\hline Problem 1 & 12 & \hspace{10mm} \hfill \\
\hline Problem 2 & 12 & \hspace{10mm} \hfill \\
\hline Problem 3 & 12 & \hspace{10mm} \hfill \\
\hline Problem 4 & 14 & \hspace{10mm} \hfill \\
\hline Total & 50 & \hspace{10mm} \hfill \\
\hline
\end{tabular}
}
\end{center}
\vfill
\begin{itemize}
\item There are 4 questions on this exam. Make sure you have all four.
\item \textbf{You must show your work on all problems.}  The correct answer
with no supporting work may result in no credit. Put a box
around your FINAL ANSWER for each problem and cross out any work
that you don't want to be graded.
\item Give exact answers, and simplify as much as possible. 
\item Use the backs of pages \textit{for scratch work only}.
\item Raise your hand if you have a question.
\item Any student found engaging in academic misconduct will receive
a score of 0 on this exam.
\item You have 50 minutes to complete the exam.  Budget your time wisely! \\
\end{itemize}
\vfill
\begin{center}GOOD LUCK!\end{center}

\newpage
\begin{enumerate} \item Let $C$ be a curve with start point $(-2,3,1)$ and end point $(3, -5, 1)$, and suppose $r(t) = (x(t), y(t), z(t))$ is a parameterization of $C$ for $t \in [0,1]$. Also, let $f: \mathbb{R}^3 \to \mathbb{R}$ be a smooth function. Assume that 

$$r(1/2) = (3,0,1), r'(1/2) = \langle 1,2,4 \rangle,$$ 

and also 

$$\nabla f(1,2,4) = \langle 2, 0, 1\rangle, \nabla f(3,0,1) = \langle -1, 0, 4 \rangle.$$ 

\begin{enumerate} \item Compute $\frac{\partial}{\partial t} f(r(t))$ at $t = 1/2$. 

\vfill 

\item What is $\frac{\partial f}{\partial y}$ at the point $(1,2,4)$?

\vfill

\end{enumerate}

\newpage

\item For each pair of conservative vector field $F$ and curve $C$, first find a potential function for $F$, and then use the fundamental theorem of line integrals to evaluate $\int_C F \cdot dr$. 

\begin{enumerate} \item[a)] $F(x,y) = (2y^2-3) \hat{i} + 4xy \hat{j}$, and $C$ is the part of the hyperbola $y = \frac{3}{x}$ between the points $(1,3)$ and $(3,1)$, traversed from left to right. 

\vfill

\item[b)] $F(x,y,z) = (y+z)\hat{i} + (x+z)\hat{j} + (x+y)\hat{k} ,$ and $C$ is the curve parameterized by $r(t) = (t, t, -1)$, for $0 \leq t \leq 4$. 

\vfill

\end{enumerate}

\newpage


\item For each of the statements below, circle \textbf{True} if you think it is true, and circle \textbf{False} if you think it is false. You may use the space to do scratch work, but no partial credit will be awarded on this question. 

\begin{enumerate} 
\item \textbf{True} \hspace{5pt} \textbf{False} \hspace{10pt} The function $f(x,y) = e^{x \sin(y)}$ satisfies $\nabla f = \sin(y) f(x,y) \hat{i} + x \cos(y) f(x,y) \hat{j}$.

\vfill

\item \textbf{True} \hspace{5pt} \textbf{False} \hspace{10pt} The vector field $F = \frac{-y \hat{i} + x \hat{j}}{x^2+y^2}$ satisfies $\int_C F \cdot dr = 0$ where $C$ is the unit circle traversed counter clockwise. 

\vfill

\item \textbf{True} \hspace{5pt} \textbf{False} \hspace{10pt} Let $G(x,y) = -4x^2 y \hat{i} - 4 x y^2 \hat{j}$. $G$ is a conservative vector field. 

\vfill

\item \textbf{True} \hspace{5pt} \textbf{False} \hspace{10pt} If $C$ is any closed curve in $\mathbb{R}^3$ (i.e. $C$ has the same start and end point) and $F$ is a smooth vector field on $\mathbb{R}^3$, then 

\begin{equation*}
\int_C F \cdot \, dr = 0.
\end{equation*}

\vfill

\item \textbf{True} \hspace{5pt} \textbf{False} \hspace{10pt} A polite employee at a phone company is called a `deferential operator.' 

\end{enumerate}

\newpage

\item Let $f:[0,1] \to [0,1]$ be any smooth function satisfying 

\begin{equation*}
f(0) = f(1) = 0.
\end{equation*}

 Let $C$ be the curve given by the graph of $f$, starting at $(1,0)$ and ending at $(0,0)$. Use Green's theorem to prove that 

\begin{equation*}
\int_0^1 x f(x) \, dx = \int_C x y \, dx + x^2 \, dy.
\end{equation*}

\textbf{Make sure to justify all your steps. }

\end{enumerate}


\end{document}
