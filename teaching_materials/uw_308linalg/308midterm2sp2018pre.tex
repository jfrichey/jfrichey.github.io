\documentclass[12 pt]{report}

\usepackage{epsfig}
\usepackage[in,empty]{myfullpage}
\usepackage{amssymb}
\usepackage{amsmath}

\baselineskip=20pt

\begin{document}

\noindent \vfill \noindent \large

\centerline{Math 308 E - Spring 2018}

\centerline{Midterm exam 2}

\centerline{Friday, May 18th, 2018}

\normalsize

\vfill
\medskip
Name: \rule{10cm}{1pt}

\bigskip

\vfill
\begin{center}
{\large
\begin{tabular}{||c|c|r||}
\hline Problem 1 & 15 & \hspace{10mm} \hfill \\
\hline Problem 2 & 10 & \hspace{10mm} \hfill \\
\hline Problem 3 & 15 & \hspace{10mm} \hfill \\
\hline Problem 4 & 10 & \hspace{10mm} \hfill \\
\hline Total & 50 & \hspace{10mm} \hfill \\
\hline
\end{tabular}
}
\end{center}
\vfill
\begin{itemize}
\item There are 4 questions on this exam. Make sure you have all four.
\item Always explain your reasoning clearly and concisely. 
\item Any student found engaging in academic misconduct will receive
a score of 0 on this exam.
\item You have 50 minutes to complete the exam. \\
\end{itemize}
\vfill
\begin{center}GOOD LUCK!\end{center}

\newpage

\begin{enumerate}

\item (15 points) Consider the matrices 

\[
A = \begin{bmatrix} 1 && 0 && 0 \\ 0 && 0 && 1 \\ 0 && 1 && 0 \end{bmatrix}, B = \begin{bmatrix} 1 && -2 && 0  \\ 2 && 3 && 4 \\ 0 && 7 && 4 \end{bmatrix}
\]

\begin{enumerate} \item (9 points) For both $A$ and $B$, find the inverse or explain why it doesn't exist. 

\vspace{8cm}

\item (3 points) Does there exist a $3 \times 3$ matrix $C$ such that det$(AC) = 8$? Give an example, or explain why this can't happen. 

\vfill

\item (3 points) Does there exist a $3 \times 3$ matrix $D$ such that det$(BD) = -1$? Give an example, or explain why this can't happen. 

\vfill

\end{enumerate} 

\newpage

\item (10 points) Consider the following set of vectors in $\mathbb{R}^4$:

\[
\mathcal{B} = \Big\{ v_1 = \begin{bmatrix} 1 \\ 0 \\ 1 \\ -1 \end{bmatrix} ,  v_2 = \begin{bmatrix} 0 \\ 0 \\ 1 \\ -1 \end{bmatrix} , v_3 = \begin{bmatrix} 0 \\ 0 \\ 1 \\ 1 \end{bmatrix}, v_4 = \begin{bmatrix} 0 \\ 2 \\ 1 \\ 1 \end{bmatrix}  \Big\}.
\]

\begin{enumerate}\item (5 points) Prove that $\mathcal{B}$ is a basis for $\mathbb{R}^4$. (Hint: it helps to show that $e_1, e_2, e_3, e_4 \in \text{span}(\mathcal{B})$.)

\vfill 

\item (5 points) Suppose $v \in \mathbb{R}^4$ is a vector satisfying $[v]_\mathcal{B} =  \begin{bmatrix} 4 \\ 2 \\ 0 \\ -1 \end{bmatrix}$. Find $v$. 

\vfill

\end{enumerate}

\newpage

\item (15 points) Let $A$ be any $2 \times 6$ matrix, and let $B$ be any $6 \times 2$ matrix. 

\begin{enumerate} \item (5 points) Is the set of vectors $v \in \mathbb{R}^2$ satisfying $(AB)^{51} v = \begin{bmatrix} 0 \\ 0 \end{bmatrix}$ a subspace of $\mathbb{R}^2$? Explain. 
\vfill
\item (5 points) Assume $AB$ is invertible (for part (b) only). Is the set of vectors $v \in \mathbb{R}^2$ satisfying $((AB)^{-1})^{3}v = \begin{bmatrix} 0 \\ -1 \\ 0 \\ 1 \end{bmatrix}$ a subspace of $\mathbb{R}^2$? Explain. 
\vfill
\item (5 points) Suppose that the nullity of $A$ is 4, and the rank of $B$ is 2. What are the dimensions of the row spaces of $A$ and $B$? 
\vfill
\item (Extra credit, 3 points) Suppose that, in addition to the assumptions in part (c),

\[
Null(A) \cap Range(B) = \{0\}.
\] 

($\cap$ is notation for the intersection of sets.) What is the rank of $AB$?
\vfill
\end{enumerate}
\newpage

\item (10 points) Let $\pi: \mathbb{R}^3 \to \mathbb{R}^3$ be the projection map

\[
\pi\Big(\begin{bmatrix} x_1 \\ x_2 \\ x_3 \end{bmatrix} \Big) = \begin{bmatrix} x_1 \\ x_2 \\ 0 \end{bmatrix},
\]

and let $R:\mathbb{R}^3 \to \mathbb{R}^3$ be the rotation map 

\[
R \Big(\begin{bmatrix} x \\ y \\ z \end{bmatrix} \Big) = \frac{\sqrt{2}}{2} \begin{bmatrix} x-y \\ x+y \\ z \end{bmatrix}.
\]

\begin{enumerate} \item (2 points) Find the matrix representing the linear transformation $T = R \circ \pi$. What is the rank of $T$?
\vfill
\item (4 points) Find a basis for the kernel of $\pi$. 
\vfill
\item (4 points) Show that the set of all vectors $v \in \mathbb{R}^3$ such that $Rv = v$ is a subspace of $\mathbb{R}^3$, and find a basis for it.
\vfill
\end{enumerate}


\end{enumerate}


\end{document}
