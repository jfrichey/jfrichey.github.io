\documentclass[11 pt]{report}

\usepackage{myfullpage}
\usepackage{amssymb}

\pagestyle{empty}

\baselineskip=20pt

\newcommand{\pp}{\par \noindent}
\begin{document}

\centerline{\bf Matrix Algebra: MATH 308 E - Spring 2018}
\vspace{0.2cm}
\begin{tabular}{lllll}
{\bf Lecturer: }    & Jacob Richey &  & {\bf Email: }       & jfrichey@math.washington.edu \\
{\bf Office: }      & Padelford C-8K        &  & {\bf Web page: }    & www.math.washington.edu/$\sim$jfrichey \\
\end{tabular}
\vspace{0.25cm}

\noindent {\bf Office Hours: Monday 10-11, Thursday 3-4.}   
I will always be in my office during office hours. If you can't make it, let me know after class or via e-mail and we can set up another time to meet.  \vspace{0.25cm}

\noindent {\bf Text:} \emph{Linear Algebra with Applications, Second Edition}, by Jeffrey Holt. Webassign is required for this course. 

\vspace{0.25cm}

\noindent {\bf Course Objectives:}  This course is motivated by the study of systems of linear equations. We will 
develop tools to solve linear problems, and develop the theory of vector spaces along the way. At first glance, solving
linear systems seems like an algebra problem -- but we will see that it has a simple and beautiful geometric interpretation. 
This course is different from the 124/5/6 sequence, in that we will rely heavily on abstract mathematical definitions and logical reasoning. 

\vspace{0.25cm}

\noindent {\bf Grading:} The weight for each part of the course is given
below. (Midterm dates are tentative.)

\vspace{-.25in}

\begin{center}\begin{tabular}{lccl} &  \\
\underline{Category} & \underline{Weight}  \\
Homework   ({\bf via Webassign})                 & 5  \\
Quizzes \& Participation             &  15  \\
Midterm 1  ({\bf Friday, April 20})                   & 25  \\
Midterm 2  ({\bf Friday, May 18})                   & 25   \\
Final Exam ({\bf Thursday, June 7, 8:30-10:20 in SMI 105})                        & 30  \\
\hline Total                                        & 100
\end{tabular}
\end{center}

\vspace{0.25cm}

\noindent {\bf Lecture:}  Lecture is on Monday, Wednesday, and Friday, 12:30-1:20 in SMI 105.  You
are responsible for all information that is discussed
during lecture. I may occasionally call on you during lecture to answer questions, so come prepared!
\vspace{0.25cm}

\noindent {\bf Homework:}  Homework will be assigned weekly through webassign. 
\vspace{0.25cm}


\noindent {\bf Exams:}  Quizzes will be given most weeks, and will be roughly 5 minutes (just 1 or 2 short questions). The midterms will be 50 minutes long andwill be given at lecture.  The final exam is cumulative. The date of the final is set by the university and is very unlikely to change under any circumstances, so you should plan your travel arrangements accordingly. During exams you are allowed one sheet of hand-written notes (8.5x11 inches, double sided), and a TI 30X IIS calculator: no other calculators are permitted. Cheating will not be tolerated. 

\vspace{.25cm}

\noindent {\bf Make-Ups:}  \emph{Late homework will not be accepted
for any reason.}  In case of observance of religious holidays or
participation in university sponsored activities, arrangements must
be made at least 1 month in advance for exams. You will be required
to provide documentation for your absence. \emph{Make-up exams will
not be given}.  If you miss an exam due to {\bf unavoidable,
compelling, and well-documented} circumstances, the other exams will
be weighted more heavily. \vspace{0.25cm}

\newpage 

\noindent {\bf Tips for succeeding in this class:}. \\
{\bf1.  Homework is crucial:} Mathematics is not a spectator sport: to learn it, you have to do it.
Reading the text and paying attention in lecture are just as important as thinking about
the material on your own. When you are stuck or confused on a problem, don't immediately check your notes or 
ask a friend to find the solution: being stuck is where the most valuable learning can occur! 

\vspace{.2cm}

\noindent Try to adopt good work habits: look at
the material within the first few days it is covered in class, so your mind can have time to ruminate on the difficult concepts.
Absorbing mathematical ideas is like eating: it is better to have one meal of math each day, rather than five in one day, so you can digest properly. If you cram too much math the night before an exam, you are bound to puke it all up the next morning.  

\vspace{.25cm}  

\noindent {\bf2.  Ask for help:} Most students will hit a wall at some point
during the course.  Some can't handle the large workload, while
others find difficulty with specific concepts in the course. When
these times arrive remember to ask for help.  Come to me, ask your classmates for help, visit the student counseling center.  These are just a
few of your options. It is better to find help earlier rather
than later. You are all smart enough to do well in this course: the
question is whether or not you are determined enough.
\vspace{0.3cm}

\noindent {\bf Resources:} \\
\noindent $\bullet$\ A link to the class website can be
found at: {\bf http://www.math.washington.edu/$\sim$jfrichey/} \\ You
will find homework assignments, review sheets, grade information, a
calendar for the term, and various bits of other useful information
there, including past exams and quizzes, TA information, etc.


\vspace{.2cm}


\noindent $\bullet$\ The University of Washington is committed to providing
access, equal opportunity and reasonable accommodation in its
services, programs, activities, education and employment for
individuals with disabilities.  To request disability accommodation
contact the Disability Services Office at least ten days in advance
at: 206-543-6450 (V), 206-543-6452 (TTY), 206-685-7264(FAX), or
dso@u.washington.edu.

\vspace{.2cm}

\noindent $\bullet$\  The Student Counseling Center provides academic skills
workshop on a variety of topics including stress management test
anxiety and time management to help you succeed at the University of
Washington. If any of these is an issue for you, check out the
schedule of workshops at {\bf
http://depts.washington.edu/scc/studyskills.html} .

\end{document}
