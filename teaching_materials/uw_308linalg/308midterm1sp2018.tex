\documentclass[12 pt]{report}

\usepackage{epsfig}
\usepackage[in,empty]{myfullpage}
\usepackage{amssymb}
\usepackage{amsmath}

\baselineskip=20pt

\begin{document}

\noindent \vfill \noindent \large

\centerline{Math 308 E - Spring 2018}

\centerline{Midterm exam 1}

\centerline{Friday, April 20th, 2018}

\normalsize

\vfill
\medskip
Name: \rule{10cm}{1pt}

\bigskip

\vfill
\begin{center}
{\large
\begin{tabular}{||c|c|r||}
\hline Problem 1 & 15 & \hspace{10mm} \hfill \\
\hline Problem 2 & 10 & \hspace{10mm} \hfill \\
\hline Problem 3 & 15 & \hspace{10mm} \hfill \\
\hline Problem 4 & 10 & \hspace{10mm} \hfill \\
\hline Total & 50 & \hspace{10mm} \hfill \\
\hline
\end{tabular}
}
\end{center}
\vfill
\begin{itemize}
\item There are 4 questions on this exam. Make sure you have all four.
\item \textbf{You must show your work on all problems.}  The correct answer
with no supporting work may result in no credit. Put a box
around your FINAL ANSWER for each problem and cross out any work
that you don't want to be graded.
\item Give exact answers, and simplify as much as possible. 
\item If you need more room, use the backs
of the pages and indicate to the grader that you have done so.
\item Raise your hand if you have a question.
\item Any student found engaging in academic misconduct will receive
a score of 0 on this exam.
\item You have 50 minutes to complete the exam.  Budget your time wisely! \\
\end{itemize}
\vfill
\begin{center}GOOD LUCK!\end{center}

\newpage

\begin{enumerate}

\item (15 points) Consider the function $T: \mathbb{R}^4 \to \mathbb{R}^2$ given by 

\[
T\Big( \begin{bmatrix} x \\ y \\ z \\ w \end{bmatrix} \Big) = \begin{bmatrix} x-2z + w \\ y + w \end{bmatrix}.
\]

\begin{enumerate} \item (5 points) Find a matrix $A$ such that $T(v) = A \cdot v$ for all $v \in \mathbb{R}^4$. 

\vfill 

\item (5 points) Is $T$ onto? Why or why not?

\vfill

\item (5 points) Is $T$ one-to-one? Why or why not?

\vfill

\end{enumerate} 

\newpage

\item (10 points) Consider the vectors $v, u, w, z \in \mathbb{R}^4$ given by 

\[
u = \begin{bmatrix} -1 \\ 1 \\ 0 \\ 2 \end{bmatrix}, \, v = \begin{bmatrix} 0 \\ -1 \\ 3 \\ 0 \end{bmatrix}, \, w = \begin{bmatrix} 0 \\ 0 \\ -1 \\ -1 \end{bmatrix}, \, z = \begin{bmatrix} 2 \\ 1 \\ -1 \\ 4 \end{bmatrix}
\]

\begin{enumerate} \item (5 points) Is $z \in span\{u, v ,w \}$? If so, find numbers $x_1, x_2, x_3$ such that $x_1 u + x_2 v + x_3 w = z$. 

\vfill 

\item (5 points) Is $\{u, v, w\}$ linearly independent? Why or why not? (Hint: use your calculation from part $a$.)

\vfill

\end{enumerate} 

\newpage

\item (15 points) Circle \textbf{True} or \textbf{False} for each of the statements below. No justification is needed. 

\begin{enumerate} \item \textbf{True} \hspace{5pt} \textbf{False} \hspace{5pt} There exist six vectors in $\mathbb{R}^5$ that are linearly independent.

\vfill 

\item \textbf{True} \hspace{5pt} \textbf{False} \hspace{5pt} A function $f: \mathbb{R} \to \mathbb{R}$ is a linear transformation if and only if $f(x) = cx$ for some $c \in \mathbb{R}$. 

\vfill 

\item \textbf{True} \hspace{5pt} \textbf{False} \hspace{5pt} A homogeneous system of equations may have infinitely many solutions. 

\vfill 

\item \textbf{True} \hspace{5pt} \textbf{False} \hspace{5pt} The matrix below is in eschelon form, and has a pivot in every column. 

\[
\begin{bmatrix} -1 && 0 && 0 \\ 0 && 1 && -1 \\ 0 && 0 && 1 \\ 0 && 0 && 0 \end{bmatrix}
\]

\vfill 

\item \textbf{True} \hspace{5pt} \textbf{False} \hspace{5pt} There exist three vectors in $\mathbb{R}^4$ that are linearly dependent. 

\vfill 

\item \textbf{True} \hspace{5pt} \textbf{False} \hspace{5pt} There exists a $3 \times 7$ matrix $A$ and a vector $b \in \mathbb{R}^3$ such that the equation $Ax = b$ has exactly three solutions for $x \in \mathbb{R}^7$. 

\vfill 

\end{enumerate}

\newpage

\item (10 points) Find an example of a matrix $A$ such that the linear transformation $T: \mathbb{R}^2 \to \mathbb{R}^3$ given by $T(v) = A \cdot v$ satisfies 

\[
T\Big( \begin{bmatrix} -1 \\ 1 \end{bmatrix}\Big) = \begin{bmatrix} 1 \\ 0 \\ 0 \end{bmatrix}, \, T\Big( \begin{bmatrix} 1 \\ 2 \end{bmatrix}\Big) = \begin{bmatrix} 1 \\ 0 \\ -1 \end{bmatrix}.
\]

How many different transformations $T$ are there that satisfy these conditions? Explain. 
\end{enumerate}

\end{document}
