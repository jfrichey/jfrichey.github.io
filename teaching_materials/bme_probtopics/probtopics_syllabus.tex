\documentclass[12pt]{article}
\usepackage{geometry}
\geometry{top=1in, bottom=1in, left=1in, right=1in}
\usepackage{amsmath, amssymb}
\usepackage{hyperref}
\usepackage{enumitem}

% Adjusting font size and spacing
\renewcommand{\baselinestretch}{1} % Adjust line spacing
\setlength{\parskip}{0.1em} % Reduce spacing between paragraphs

\begin{document}

\begin{center} {\bf{ Topics in Probability -- A Walk in the Random Park}} \hfill (Spring 2025)\\
\end{center}

\noindent{\bf{Course time:}} Thursday, 16:15 - 17:45 \\
{\bf{Location:}} BME Building H, Room H406 \\
{\bf{Instructor: }} Jacob Richey \hfill {\bf{Email: }} jrichey@renyi.hu \\
{\bf{Neptun code: }} BMETESZMsMVFVA-00 \\

%{\bf{Office: }} Rényi Institute 308 \hfill {\bf{Office hours:}} ?? \\


\noindent \textbf{Suggested texts}
\begin{itemize} \item {\em{Random Walk: A Modern Introduction,}} Gregory F. Lawler and Vlada Limic. Cambridge University Press, 2010. 
\item {\em{Percolation,}} Geoffrey Grimmett. Springer, 1999.
\item {\em{50 Years of First-Passage Percolation,}} Antonio Auffinger, Michael Damron and Jack Hanson. American Mathematical Society, 2017. 
\end{itemize}


\noindent \textbf{Prerequisites:} Knowledge of basic facts from calculus and probability will be assumed. We will build the remaining mathematical tools needed as we go along. 

\vspace{10pt}

% Syllabus Section
\noindent \textbf{Course outline}: This course will survey results and theories related to fundamental models originating in pure probability and physics, with a focus on simple random walks, percolation theory, and first passage percolation. Questions and ideas related to these models have guided research advances throughout the 20th century, and recent international prizes have been awarded to scholars who touched these problems, including Duminil-Copin, Furstenberg, Kesten, Lawler, and Talagrand. The product has been new and elegant mathematics -- including potential theory, correlation inequalities, and Busemann functions, to name a few -- though many challenges still remain today. 

\vspace{10pt}

% Course Schedule Section
\noindent \textbf{Syllabus}
\begin{enumerate} \item Simple random walks: Review of LLN, CLT, Polya's theorem; LCLT, potential theory, intersections of random walks. Possible extra topics: electrical networks, hitting times, cut times, self-avoiding walks, loop erased random walks, uniform spanning forest. 
	
\item Percolation theory: Phase transition and critical value, Russo formula, uniqueness of the infinite cluster. Possible extra topics: correlation inequalities, crossing probabilities, critical exponents, dynamical percolation. 

\item First passage percolation: Subadditivity and LLN, Shape theorem, Variance and fluctuations of passage times. Possible extra topics: properties of geodesics, Busemann functions, dynamical FPP. 

\end{enumerate}

% Grading Section
\noindent \textbf{Grading:} Students taking the course for credit will be expected to contribute by presenting a special topic, academic paper, or problem set solutions in class. 



\end{document}